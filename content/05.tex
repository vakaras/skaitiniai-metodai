\Chapter{Splainai}

% 2014-03-04

\TODO{\slide{2}}

%Priminimas: Interpoliavimo uždaviniai – skaitymas tarp matematinių
%lentelių eilučių.

%Anksčiau būdavo sprendžiami tik tokio tipo uždaviniai.

%Istorinis nukrypimas: Po revoliucijos Prancuzijoje, kas liko be darbo?
%Kirpėjai. Todėl jie buvo įdarbinti sudarinėti matematinių lentelių.

%Kitas atvejis: Kai JAV buvo didžioji depresija, tai bedarbiai buvo
%įdarbinti sudarinėti matematinių lentelių.

\TODO{\slide{3}}

Interpoliavimas daugianariais netinka (atsiranda osciliacijos):
\begin{itemize}
  \item Duomenys su dideliu gradientu (lygus grafikas su staigiu piku).
  \item Duomenys su triukšmu.
  \item Neglodus paviršius (netolydi išvestinė).
\end{itemize}

\TODO{\slide{4}}

\TODO{\slide{5}}

Interpoliavimas daugianariais:
\begin{itemize}
  \item $L_{n}(x)$ eina per visus duotus taškus.
  \item Jei atsiranda papildomi duomenys, reikia didinti daugiaunario laipsnį,
    kas nevisada tikslinga, nes gali atsirasti osciliacijos jei tiriama
    funkcija nepanaši į daugianarį.
\end{itemize}

Alternatyva: interpoliavimas dalimis daugianarėmis funkcijomis.

\TODO{\slide{6}}

\TODO{\slide{7}}

Tarkime, kad funkcija $y=f(x)$ yra apibrėžta reikšmių lentele
$(x_{i};y_{i}), i=0,1,\ldots,N$. Intervale sudarome jos interpoliacinę
funkciją iš neaukšto, pirmojo arba antrojo laipsnio interpoliacinių
daugianarių, apibrėžtų atitinkamoje intervalo $\left[ x_{0}; x_{N} \right]$
dalyje. Gauta funkcija yra tolydi visame intervale, net nėra
diferencijuojama interpoliavimo taškuose.

\TODO{\slide{8}}

Dalimis tiesinė interpoliacinė funkcija:
\begin{align*}
  L_{0}^{1}(x) &= a_{0}(x-x_{0}) + b_{0} & x_{0} \leq x \leq x_{1} \\
  L_{1}^{1}(x) &= a_{1}(x-x_{1}) + b_{1} & x_{1} \leq x \leq x_{2} \\
               & \vdots \\
  L_{i}^{1}(x) &= a_{i}(x-x_{i}) + b_{i} & x_{i-1} \leq x \leq x_{i} \\
               & \vdots \\
  L_{N-1}^{1}(x) &= a_{N-1}(x-x_{N-1}) + b_{N-1} & x_{N-1} \leq x \leq x_{N} \\
\end{align*}
čia:
\begin{align*}
  a_{i} &= \frac{y_{i+1} - y_{i}}{x_{i+1} - x_{i}}, \\
  b_{i} &= y_{i}.
\end{align*}

\TODO{\slide{9}}

%Dalimis tiesinė – du taškus sujungiame tiesėmis.

\TODO{\slide{10}}

Interpoliavimas splainais:
\begin{itemize}
  \item Nedidelio laipsnio daugianariai jungia duotuosius taškus.
  \item Funkcija glodi visame intervale (taip pat ir vidiniuose interpoliavimo
    mazguose).
  \item Nedidelio laipsnio daugianariai neleidžia atsirasti osciliacijoms.
\end{itemize}

%Idėja – bandyti „suklijuoti“ gražiai. Tai vadinama
%interpoliavimu splainais. (Jungiam taškus daugianariais.)

%Ne tik funkcija, bet ir jos išvestinės turi būti tolydžios.

\TODO{\slide{11}}

%\en{spline} – lazdelė.

%Idėja atėjo ne iš matematikos, bet iš inžinerijos. Kaip senovėje braižinių
%darytojai brėždavo sudėtingas kreives: fiksuojame taškus adatėlėmis, o
%juos sujungiame juostele (spline).

\TODO{\slide{12-14}}

\TODO{\slide{15}}

%Nuo tokių uždavinių viskas prasidėjo 20 amžiaus viduryje.

%Laivui svarbu ne tik dinamika, bet ir statika. Yra pagrindinės lygtis,
%o kas tarp jų? Tarp jų kaip tik ir buvo taikomi splainai.

\section{Tiesinis splainas}

\TODO{\slide{16}}

\begin{equation*}
  s_{i}^{1}(x) = a_{i}x + b_{i}
\end{equation*}

\TODO{\slide{17}}
\TODO{\slide{18}}

%Tiesiniam splainui tolydumo reikalavimas yra tik pačiai funkcijai, bet ne
%jos išvestinei.

\TODO{\slide{19}}

\begin{equation*}
  S^{1} = 
  \begin{cases}
    f(x_{0}) + f(x_{0},x_{1})(x-x_{0}), & x_{0} \leq x \leq x_{1}; \\
    f(x_{1}) + f(x_{1},x_{2})(x-x_{1}), & x_{1} \leq x \leq x_{2}; \\
    \vdots & \vdots \\
    f(x_{N-1}) + f(x_{N-1},x_{N})(x-x_{N-1}), & x_{N-1} \leq x \leq x_{N}; \\
  \end{cases}
\end{equation*}
Tiesinis splainas sutampa su dalimis tiesine interpoliacine funkcija.

%Iš sudarymo išplaukia, kad egzistuoja vienintelis sprendinys.

\TODO{\slide{20}}

%Kiekvienam intervalui gauname po tiesę.

\TODO{\slide{21}}

Tiesinio splaino trūkumas – nėra glodumo interpoliavimo mazguose
(pirmoji išvestinė netolydi).

Tolydžiosios išvestinės gaunamos taikant aukštesnės eilės splainus:
\begin{itemize}
  \item Vidiniuose mazguose splainas yra tolydi funkcija:
    \begin{equation*}
      s_{i-1}(x_{i}) = S_{i}(x_{i}), i=1,\ldots,N-1.
    \end{equation*}
  \item Vidiniuose mazguose išvestinė yra tolydi:
    \begin{equation*}
      S_{i-1}^{(k)}(x_{i}) = S_{i}^{(k)}(x_{i}), i=1,\ldots,N-1, k=1,\ldots,m-1.
    \end{equation*}
\end{itemize}

\section{Kvadratinis splainas}

\TODO{\slide{22}}

Kvadratinis splainas: pirma išvestinė tolydi, bet antra išvestinė gali būti
netolydi.

\TODO{\slide{23}}

\TODO{\slide{24}}

Kvadratinio splaino sudarymas:
\begin{enumerate}
  \item Interpoliavimo ir tolydumo sąlygos ($2N$ lygčių):
    \begin{align*}
      S_{i}(x_{i}) &= y_{i}, & i = 0,1,\ldots,N-1,\\
      S_{i}(x_{i+1}) &= y_{i+1} & i = 0,1,\ldots,N-1.\\
    \end{align*}
  \item Išvestinių tolydumo sąlygos vidiniuose taškuose ($N-1$ lygtis):
    \begin{align*}
      S'_{i-1}(x_{i}) &= S'_{i}(x_{i}), & i = 1,\ldots,N-1.\\
    \end{align*}
  \item Papildoma sąlyga (duota išvestinė viename iš kraštinių taškų):
    \begin{equation*}
      e_{0} = 0
    \end{equation*}
    Ši sąlyga vadinama natūraliąja.
\end{enumerate}

\TODO{\slide{25}}
Gautos lygtys:
\begin{align*}
  \t{Iš interpoliavimo ir tolydumo sąlygos:} \\
  a_{i}x_{i}^{2} + b_{i}x_{i} + c_{i} &= y_{i}, & i = 0,1,\ldots,N-1, \\
  a_{i}x_{i+1}^{2} + b_{i}x_{i+1} + c_{i} &= y_{i+1}, & i = 0,1,\ldots,N-1 \\
  \t{Iš išvestinių tolydumo sąlygos:} \\
  2a_{i-1}x_{i} + b_{i-1} &= 2a_{i}x_{i} + b_{i}, & i = 1,\ldots,N-1 \\
  \t{Papildoma sąlyga:} \\
  2a_{0}x_{0} + b_{0} &= e_{0}.
\end{align*}

\TODO{\slide{26}}

Lygčių sistemą galime spręsti skaidydami į blokus.

\TODO{\slide{27}}

Algoritmas:
\begin{enumerate}
  \item $e_{0}$ duotas, sprendžiame:
    \begin{equation*}
      \begin{pmatrix}
        x_{0}^{2} & x_{0} & 1 \\
        x_{1}^{2} & x_{1} & 1 \\
        2x_{0} & 1 & 0 \\
      \end{pmatrix}
      \begin{pmatrix}
        a_{0} \\
        b_{0} \\
        c_{0} \\
      \end{pmatrix}
      =
      \begin{pmatrix}
        y_{0} \\
        y_{1} \\
        e_{0} \\
      \end{pmatrix}
    \end{equation*}
  \item $e_{1} = 2a_{1}x_{1} + b_{1}$ iš suderinamumo sąlygų:
    \begin{align*}
      S'_{0}(x_{1}) &= S'_{1}(x_{1}) \\
      \underbrace{2a_{0}x_{1} + b_{0}}_{e_{1}} &= 2a_{1}x_{1} + b_{1} \\
    \end{align*}
    $x_{1}$ žinom, o $a_{0}$ ir $b_{0}$ apsiskaičiavome ankstesniame
    žingsnyje. Taigi, $e_{1}$ žinome ir galime spręsti kitą bloką:
    \begin{equation*}
      \begin{pmatrix}
        x_{1}^{2} & x_{1} & 1 \\
        x_{2}^{2} & x_{2} & 1 \\
        2x_{1} & 1 & 0 \\
      \end{pmatrix}
      \begin{pmatrix}
        a_{1} \\
        b_{1} \\
        c_{1} \\
      \end{pmatrix}
      =
      \begin{pmatrix}
        y_{1} \\
        y_{2} \\
        e_{1} \\
      \end{pmatrix}
    \end{equation*}
  \item Ir t.t.
\end{enumerate}

\TODO{\slide{28}}

\section{Kubinis splainas}

\TODO{\slide{29}}

\begin{equation*}
  S_{i}^{3}(x) = a_{i}x^{3} + b_{i}x^{2} + c_{i}x + d_{i},
    x_{i} \leq x \leq x_{i+1}.
\end{equation*}

Kaip ir kvadratiniam splainui gaunama $4N$ koeficientų ir $4N$ lygčių
sistema.

\begin{itemize}
  \item Vidiniuose mazguose tolydumas.
  \item Intervalo galai fiksuoti.
  \item Vidiniuose mazguose išvestinės tolydžios.
  \item Papildomos sąlygos antrosios eilės išvestinėms intervalo galuose.
\end{itemize}

%Viskas yra analogiškai, tik funkcija yra kubinė parabolė.

%Intervalo galai fiksuoti – galima interpretuoti gana įvariai. Mes
%užfiksuojame duodami papildomas sąlygas.

\TODO{\slide{30}}
\TODO{\slide{31}}

%Reikalaujame, kad vidiniuose taškuose būtų tolydu. Mums trūksta
%dviejų lygčių, įsivedama natūraliųjų kraštinių sąlygas.
%\note{Kubinis splainas, gali turėti ir kitokias papildomas sąlygas,
%nebūtinai natūraliasias.}

\TODO{\slide{32}}
\TODO{\slide{33}}

Pažymėkime $g_{i} = S''_{i}(x_{i})$. Susirandame tiesinį splainą ir
jį integruojame du kartus. Norėdami išspręsti lygtį įsistatome
kitą žinomą tašką. Tikslas: surasti konstruktyvų algoritmą, kaip
apskaičiuoti $g$.

\TODO{\slide{34}}

Sugrupavus $g$ atžvilgiu, gauname triįstrižainę lygčių sistemą.

\TODO{\slide{35}}

Triįstrižainė lygčių sistema $g$ atžvilgiu:
\begin{equation*}
  h_{i-1}g_{i-1} + 2(h_{i-1}+h_{i})g_{i} + h_{i}g_{i+1}
    = 6\left(
        \frac{y_{i+1} - y_{i}}{h_{i}}
        - \frac{y_{i}-y_{i-1}}{h_{i-1}} \right),
\end{equation*}
čia:
\begin{align*}
  i &= 1,\ldots,N-1, \\
  g_{0} &= 0, \\
  g_{N} &= 0, \\
  h_{i} &= x_{i+1} - x_{i}, \\
  f_{i} &= \frac{y_{i+1} - y_{i}}{h_{i}}. \\
\end{align*}

\TODO{\slide{36}}

Nereikia tikrinti triįstrižainės matricos konvergavimo sąlygos, nes
turime $2 > 1$ (tai yra, ji visada galioja).

\TODO{\slide{37}}

Kubinio splaino lygtis:
\begin{equation*}
  S^{3}_{i}(x)
    = y_{i}
    + e_{i}(x-x_{i})
    + G_{i}(x-x_{i})^{2}
    + H_{i}(x-x_{i})^{3},
    i = 0,1,\ldots,N-1.
\end{equation*}
Čia:
\begin{align*}
  e_{i}
    &= \frac{y_{i+1}-y_{i}}{h_{i}}
    - g_{i+1}\frac{h_{i}}{6}
    - g_{i}\frac{h_{i}}{3}, \\
  G_{i} &= \frac{g_{i}}{2}, \\
  H_{i} &= \frac{g_{i+1} - g_{i}}{6h_{i}}. \\
\end{align*}


\TODO{\slide{38}}
\begin{prop}[Kubinio splaino paklaida]
  Jei funkcija $y(x) \in C^{4}\left[ a; b \right]$, tai kubinio splaino
  $S^{3}(x)$ paklaida ir jo išvestinių ${S^{3}}'(x)$, ${S^{3}}''(x)$ paklaida
  įvertinama nelygybe:
  \begin{equation*}
    |y^{(k)}(x) - S^{3(k)}(x)|
      \leq M_{4}h^{2-k}(h^{2}
      + \max(|y''(a) - {S^{3}}''(a)|, |y''(b)-{S^{3}}''(b)|)),
    k = 0,1,2
  \end{equation*}
  \note{Kai $k=0$, tai vertiname pačios funkcijos paklaidą. Kai $k=1$, tai
  – pirmos išvestinės.}
\end{prop}
%Jeigu išvestines žinome tiksliai, tai kažkur bus 0. Kai tiksliai
%aproksimuojame taškus, tik tada tiksliai galime aproksimuoti. Kitu
%atveju, mums paklaida maišo.
Kai $y''(a)={S^{3}}''(a)$, $y''(b)={S^{3}}''(b)$, tai paklaidą galima vertinti:
\begin{equation*}
  |y^{(k)}(x) - S^{3(k)}(x)| \leq M_{4}h^{4-k}, k = 0, 1, 2
\end{equation*}
Kai $y''(a) \neq {S^{3}}''(a)$, $y''(b) \neq {S^{3}}''(b)$, tai paklaidą
galima vertinti:
\begin{equation*}
  |y^{(k)}(x) - S^{3(k)}(x)| \leq M_{4}h^{2-k}, k = 0, 1, 2
\end{equation*}

\TODO{\slide{39-42}}
\TODO{\slide{43-47}}

%Išlenkimai priklauso nuo kraštinių sąlygų.

\section{Kraštinės sąlygos}

\TODO{\slide{48-50}}
\TODO{\slide{51}}

%Splainas ne visada išlaiko monotoniškumą.

\section{Splainų taikymai}

\TODO{\slide{52}}
\TODO{\slide{53}}

Renkantis metodą reikia atsižvelgti į specifiką. Pavyzdžiui, medicinoje
yra reikalavimas neprikurti to ko ten nėra, todėl dažnai yra naudojamas
artimiausio kaimyno interpoliavimas.

\TODO{\slide{54}}

Su daugiamačiu interpoliavimu yra problema kaip atvaizduoti duomenis.
Pavyzdžiui, jei turime kambario temperatūros funkciją. Galimas
variantas yra žaisti spalvomis.

\TODO{\slide{55}}
\TODO{\slide{56}}

%Duomenys yra tikri: kiaulės kraujagyslių tamprumo tikrinimas. Tempiame
%kraujagyslę ir tikriname kada plyš.

%\section{Splainų interpoliavimas}
%\cite[170-173]{textbook}
%\section{12p}
%\cite[171]{textbook}
%\section{Kubinis splainas}
%\cite[174-175]{textbook}
