\Chapter{Kompiuterių aritmetika ir algoritmai}
% 2014-02-04

Šio kurso pavadinimas „Skaitiniai metodai“ yra angliško termino
„Scientific Computing” atitikmuo. Skaitiniais metodais yra
sprendžiami matematinio modeliavimo uždaviniai, kurie yra užrašyti
algebrinėmis formulėmis (kurias galima įvykdyti kompiuteriu).

Šiuo kursu siekiama išmokti:
\begin{enumerate}
  \item suformuluoti sprendimo metodiką;
  \item įvertinti metodo tinkamumą, jo privalumus ir trūkumus\footnote{Yra
    labai svarbu sugebėti įvertinti paklaidas. Tai yra, atsakyti
    į klausimą su kokia paklaida buvo išspręstas uždavinys.}
\end{enumerate}
\begin{note}
  \TODO{Išversti.}
  Simulation techniques aim at:
  \begin{itemize}
    \item Solving the right equations!
    \item Solving the equations right!
    \item Solving the equations fast!
  \end{itemize}
\end{note}

\TODO{\slide{3-6}}

\TODO{\slide{7-8}}
Ketvirtosios eilės daugianaris yra aukščiausios eilės daugianaris,
kurio sprendinį visada galima rasti analiziniais metodais (rąstas
sprendinys būtų tikslus). Taigi ketvirtosios eilės daugianario
tikslių šaknų radimo uždavinys iš esmės skiriasi nuo penktos
eilės daugianario šaknų radimo uždavinio. Jei norime rasti tik apytiksles
šaknis, tai esminių skirtumų tarp šių uždavinių nebelieka.

Pavyzdžiai problemų, kurios gali būti išspręstos panaudojant \emph{baigtinį skaičių
elementariųjų operacijų}:
\begin{enumerate}
  \item ketvirto ir žemesnio laipsnio daugianario šaknies radimas;
  \item tiesinių lygčių sistemos;
  \item tiesinis optimizavimas;
  \item keliaujančio pirklio uždavinys.
\end{enumerate}
Pavyzdžiai problemų, kurios negali būti išspręstos šiuo būdu:
\begin{enumerate}
  \item tikrinės reikšmės radimas $n \times n$ matricai;
  \item kelių kintamųjų funkcijos minimizavimas;
  \item integralo apskaičiavimas;
  \item paprastosios diferencialinės lygties sprendimas;
  \item dalinių išvestinių lygties sprendimas.
\end{enumerate}

\begin{note}
  Tai, kad uždavinys priklauso pirmajai grupei, dar nereiškia, kad jis yra
  lengvesnis. Pavyzdžiui, algoritmas randantis tikslų keliaujančio pirklio
  uždavinio sprendinį yra nepritaikomas, kai $N$ yra didelis.
\end{note}

\TODO{\slide{9}}

\TODO{\slide{10}}
\begin{defn}[Matematinis modeliavimas]
  Taikomosios matematikos dalis, skirta įvairių sričių (fizikinių, biologinių,
  cheminių, ekonominių ir t.t.) uždavinių sprendimui naudojant virtualiojo
  eksperimento metodiką.
\end{defn}
Galimi uždavinio sprendimo įrankiai:
\begin{itemize}
  \item analiziniai metodai – galima rasti tikslų sprendinį;
  \item artutiniai metodai – ieškoma artimų sprendinių;
  \item skaitiniai metodai – tolydų keičiame diskrečiu;
  \item statistiniai metodai;
  \item grafikai;
  \item t. t.
\end{itemize}

\TODO{\slide{11-12}}
Matematinių uždavinių sprendimo etapai:
\begin{enumerate}
  \item Matematinis modeliavimas yra naudojamas sprendžiant
    taikomuosius arba fizikinius uždavinius. Sudarant matematinį
    modelį yra siekiama supaprastinti realų reiškinį atsisakant
    nereikalingų dalių, bet paliekant visas esmines. Sudarant modelį
    turi dalyvauti visų sričių specialistai.
  \item Parenkame skaitinį metodą.
  \item Suprogramuojame.
  \item Tikriname gautus rezultatus ir jei kažkas blogai, grįžtame atgal.
\end{enumerate}

\TODO{\slide{13-16}}
Renkantis skaitinį metodą yra atsižvelgiama į šias jų savybės:
\begin{enumerate}
  \item konvergavimas į sprendinį – kuo daugiau taškų imame, tuo sprendinys
    turi būti arčiau tikrojo;
  \item konservatyvumas;
  \item korektiškumas;
  \item realizavimo galimybės.
\end{enumerate}
Algoritmas turi būti stabilus: jei pradinės sąlygos arti, tai ir rezultatas
turi būti arti.

\TODO{\slide{17-20}}
Paklaidų šaltiniai ir klasifikacija:
\begin{enumerate}
  \item Matematinio modelio paklaida. Atsiranda dėl atmestų faktorių.
  \item Metodo paklaida. Atsiranda dėl įtrauktų į modelį faktorių.
  \item Apvalinimo paklaida. Atsiranda dėl uždavinio sąlygotumo, jautrumo
    ir algoritmo stabilumo.
\end{enumerate}

\TODO{\slide{21-22}}
Matematinio modelio paklaida (dėl atmestų faktorių):
\begin{itemize}
  \item netikslus uždavinio matematinis aprašymas;
  \item duomenų paklaida.
\end{itemize}
Ši paklaida dar vadinama nepašalinamąja paklaida. Tokios paklaidos
atsiranda, nes pereinam nuo konkretaus uždavinio prie modelio. Šiame
kurse modelio paklaidos nėra nagrinėjamos, nes mes su jomis nelabai
ką galime padaryti. Nagrinėjame skaičiavimo metodų paklaidas.

\TODO{\slide{23}}
Metodo paklaida (dėl įtrauktų į modelį faktorių). Metodo paklaida
atsiranda dėl perėjimo nuo tolydžios sistemos prie diskrečios.
Metodo paklaidas galime valdyti, tačiau nėra prasmės skaičiuoti
tiksliau nei modelio paklaida.

\TODO{\slide{24}}
Apvalinimo paklaida (uždavinio sąlygotumas, jautrumas, algoritmo stabilumas):
\begin{itemize}
  \item įvedant duomenis;
  \item atliekant aritmetinius veiksmus;
  \item išvedant duomenis.
\end{itemize}
Kitaip dar vadiname skaičiuojamąja paklaida.

\TODO{\slide{25-28}}
\begin{prop}
  Dviejų apytikslių skaičių sumos ar skirtumo absoliučioji paklaida yra ne
  didesnė už tų skaičių absoliučiųjų paklaidų sumą.
\end{prop}
\TODO{\slide{29-30}}
\begin{prop}
  Dviejų apytikslių skaičių sandaugos ar dalmens santykinė paklaida yra ne
  didesnė už tų skaičių santykinių paklaidų sumą.
\end{prop}

\TODO{\slide{29-30}}
Paklaidos taip pat atsiranda dėl to, kaip kompiuteriai saugo
realiuosius skaičius.

\TODO{\slide{31-38}}
Siekiant išvengti didelių skaičiavimo paklaidų, algoritmą reikia sudaryti
taip, kad:
\begin{enumerate}
  \item Nebūtų atimami dideli (palyginus su gautu skirtumu) artimi skaičiai.
  \item Nebūtų dalybos iš mažo skaičiaus.
  \item Skaičiai būtų sudedami jų didėjimo tvarka.
\end{enumerate}

\TODO{\slide{39-44}}
