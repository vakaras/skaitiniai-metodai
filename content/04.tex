\Chapter{Interpoliavimas}

% 2014-02-25

\section{Duomenų aproksimavimas}

\TODO{\slide{2}}

Tegul funkcijos $f$ reikšmės žinomos tik kai kuriuose pasirinktuose
taškuose (pavyzdžiui, tai yra matavimo rezultatai). Tikslas yra
sukonstruoti tolydžią funkciją, kuri leidžia apytiksliai nustatyti
funkcijos $f$ tarpines reikšmes iš žinomų jos reikšmių, arba
aproksimuoti funkciją $f$ kita paprastesnio pavidalo funkcija $g$, kuri
naudojama kaip jos pakaitalas, pavyzdžiui, integravimo uždaviniuose.

Metodai:
\begin{itemize}
  \item Mažiausių kvadratų metodas – minimizuojamas atstumas nuo
    duotųjų taškų iki kreivės. Jį patartina naudoti, kai pradiniai
    taškai yra apytiksliai (gali turėti matavimo paklaidą).
  \item Interpoliavimas funkcijomis – kreivė eina tiksliai per duotus
    taškus.
\end{itemize}

\TODO{\slide{3}}

Aproksimuojančios kreivės:
\begin{itemize}
  \item Regresija (mažiausių kvadratų metodas).
  \item Tiesinis interpoliavimas – taškai sujungiami tiesiomis atkarpomis.
  \item Interpoliavimas splainais – funkcija ir jos išvestinė yra tolydžios.
\end{itemize}

\TODO{\slide{4-7}}

\TODO{\slide{8}}

Bitiesinis interpoliavimas: imame informaciją iš virš, žemiau,
kairiau, dešiniau esančių pikselių.

Bikubinis interpoliavimas: imame informaciją papildomai iš virš kairiau, virš,
virš dešiniau, … esančių pikselių.

\TODO{\slide{9}}

%Toliau žymėjimas skaidrėse: $y$ reiškia tą patį kaip ir $f(x)$.

Interpoliavimas daugianariais:
\begin{enumerate}
  \item Atkarpos jungia du taškus.
  \item Kvadratinė parabolė eina per tris taškus.
  \item Kubinė parabolė eina per keturis taškus.
\end{enumerate}

\TODO{\slide{10}}

%$n$ – daugianario laipsnis, 1, 2 arba 3.

%$L_{n}(x_{0}) = f(x_{0})$  – sudaryta funkcija turi įgyti žinomas reikšmes
%sąlygose duotuose taškuose.

Jei yra žinoma $n+1$ funkcijos $f(x)$ reikšmė, tai galima sudaryti $n$-ojo
laipsnio interpoliacinį daugianarį:
\begin{equation*}
  L_{n}(x) = a_{n}x^{n} + a_{n-1}x^{n-1} + \cdots + a_{1}x + x_{0}.
\end{equation*}

\TODO{\slide{11}}

%Apsimoka tik tada, kai n nedidesnis nei 5 arba 6; arba kai žinome, kad ieškome
%daugianario.

\TODO{\slide{12}}

%Iš interpoliavimo sąlygų gauname lygčių sistemą.

%Galime išspręsti lygčių sistemą tik tada kai determinantas 0.

%Kadangi pas mus uždavinys sudarytas taip, kad x yra skirtingi, tai iš to
%išplaukia, kad egzistuoja vienintelis sprendinys.

\section{Tiesinis interpoliavimas}

\cite[158-163]{textbook}

\TODO{\slide{13}}

\begin{defn}[Tiesinis interpoliacinis daugianaris]
  \begin{equation*}
    L_{1}(x) = f(x_{1}) + \frac{f(x_{2} - f(x_{1}))}{x_{2} - x_{1}}(x - x_{1})
  \end{equation*}
\end{defn}

\TODO{\slide{14}}

%Pavyzdys: tiriama funkcija $e^{x}$.

\TODO{\slide{15}}

%Žalia – kai interpoliavimui naudojame taškus $(1, e^{1})$ ir $(5, e^{5})$.
%Raudona – kai interpoliavimui naudojame taškus
% $(1,5; e^{1,5})$ ir $(2,5; e^{2,5})$

\TODO{\slide{16}}

\note{Susitarimas: Jeigu mes vertiname antrą išvestinę, tai konstanta
imame $M_{2}$, jei trečios, tai $M_{3}$ ir t. t.}

\begin{prop}[Paklaidos įvertis]
  Jei funkcijos $f(x)$ antroji išvestinė intervale
  $\left[ x_{i}; x_{i+1} \right]$ yra aprėžta, tai yra:
  \begin{equation*}
    |f''(x)| \leq M_{2}, \t{ kai } x_{i} < x < x_{i+1},
  \end{equation*}
  tai tiesinio interpoliavimo paklaida įvertinama nelygybe:
  \begin{equation*}
    |f(x) - L_{1}(x)| \leq \frac{1}{2}M_{2}|(x-x_{i})(x-x_{i+1})|
    \leq \frac{1}{2}M_{2}h_{i}^{2},
  \end{equation*}
  čia:
  \begin{align*}
    h_{i} &= x_{i+1} - x_{i}, \\
    x \in \left[ x_{i}; x_{i+1} \right].
  \end{align*}
\end{prop}

\TODO{\slide{17-19}}

%Esmė: įrodėme bet kokiam taškui, tai iš to išplaukia, kad visiems taškams.

\section{Skirtumų santykio formulės ir skirtumų lentelės sudarymas}

\cite[163]{textbook}

\TODO{\slide{20}}

Kadangi $f(x)$ nėra žinoma, tai tiksli konstanta $M_{2}$ įvertyje
\begin{equation*}
  |f''(x)| \leq M_{2}
\end{equation*}
irgi nėra žinoma. Todėl reikia turėti bent apytikslę antros eilės išvestinės
reikšmę.

\TODO{\slide{21}}

\begin{defn}[Pirmosios eilės skirtumų santykis]
  \begin{equation*}
    f(x_{i}, x_{i+1}) = \frac{f(x_{i+1}) - f(x_{i})}{x_{i+1} - x_{i}}
  \end{equation*}
  \note{Pirmosios eilės skirtumų santykis yra išvestinės artinys.}
\end{defn}

\begin{defn}[Antrosios eilės skirtumų santykis]
  \begin{equation*}
    f(x_{i},x_{i+1},x_{i+2})
      = \frac{f(x_{i+1},x_{i+2})-f(x_{i},x_{i+1})}{x_{i+2}-x_{i}}
  \end{equation*}
  \note{Antrosios eilės skirtumų santykis yra antros išvestinės artinys.}
\end{defn}

\begin{defn}[$n$-osios eilės skirtumų santykis]
  \begin{equation*}
    f(x_{0},x_{1},\ldots,x_{n})
      = \frac{f(x_{1},\ldots,x_{n})-f(x_{0},\ldots,x_{n-1})}{x_{n}-x_{0}}
  \end{equation*}
  \note{$n$-osios eilės skirtumų santykis yra $n$-osios išvestinės artinys.}
\end{defn}

\TODO{\slide{22}}

\TODO{\slide{23}}

Paklaidos įverčio formulėje vietoj $M_{2}$ galima naudoti apytikslį antrosios
išvestinės rėžį:
\begin{equation*}
  f''(x) \approx 2f(x_{i}, x_{i+1}, x_{i+2}).
\end{equation*}

Bendruoju atveju:
\begin{equation*}
  f^{(n)}(x) \approx n!f(x_{0},x_{1},\ldots,x_{n}).
\end{equation*}

\section{Kvadratinis interpoliavimas}

\cite[165-167]{textbook}

\TODO{\slide{24}}

Kvadratiniam interpoliavimui reikalingi trys interpoliavimo mazgai.
Taškai yra jungiami parabole.
\begin{equation*}
  L_{2}(x) = b_{1} + b_{2}(x-x_{1}) + b_{3}(x-x_{1})(x - x_{2})
\end{equation*}

%Raudona – nežinoma funkcija.
%Mėlyna – parabolė.
%$L_{2}(x)$ skiriasi nuo $L_{1}$ tuo, kad pridėjome vieną papildomą narį.

\TODO{\slide{25}}

\TODO{\slide{26}}
Kvadratinio interpoliavimo koeficientų radimas:
\begin{equation*}
  L_{2}(x)
    = f(x_{1})
    + f(x_{1},x_{2})(x-x_{1})
    + f(x_{1},x_{2},x_{3})(x-x_{1})(x-x_{2})
\end{equation*}

\TODO{\slide{27-29}}

\TODO{\slide{30}}
\begin{prop}
  Jei funkcijos $f(x)$ trečiosios eilės išvestinė intervale
  $\left[ x_{i}; x_{i+2} \right]$ yra apibrėžta:
  \begin{equation*}
    |f'''(x)| \leq M_{3}, \t{ kai } x_{i} < x < x_{i+2},
  \end{equation*}
  tai kvadratinio interpoliavimo paklaida įvertinama nelygybe:
  \begin{equation*}
    |f(x) - L_{2}(x)|
      \leq \frac{1}{6}M_{3}|(x-x_{i})(x-x_{i+1})(x-x_{i+2})|
      \leq \frac{\sqrt(3)}{27}M_{3}h^{3},
  \end{equation*}
  čia:
  \begin{align*}
    h &= \max \left\{ x_{i+1} - x_{i}; x_{i+2} - x_{i+1} \right\} \\
    x &\in \left[ x_{i}; x_{i+2} \right] \\
  \end{align*}
\end{prop}

\TODO{\slide{31}}

\section{Niutono interpoliacinis daugianaris}

\cite[168-169]{textbook}

\TODO{\slide{32}}

Niutono interpoliacinė formulė yra tiesinio ir kvadratinio interpoliavimo
formulių apibendrinimas bet kokiam $n$.

\begin{defn}[Niutono interpoliacinė formulė]
  \begin{align*}
    L_{n}(x)
      &= f(x_{0}) \\
      &+ f(x_{0},x_{1})(x - x_{0}) \\
      &+ f(x_{0},x_{1},x_{2})(x-x_{0})(x-x_{1}) \\
      &+ \cdots \\
      &+ f(x_{0},x_{1},\ldots,x_{n})(x-x_{0})(x-x_{1})\cdots(x-x_{n-1}) \\
  \end{align*}
\end{defn}
\begin{itemize}
  \item Jei yra žinoma užtektinai funkcijos $f(x)$ reikšmių, tai šią formulę
    legvai galima papildyti naujais nariais ir kartu padidinti jos „tikslumą“.
  \item Interpoliavimo taškai $x_{i}$ gali būti pasiskirstę kaip tolygiai,
    taip ir netolygiai, juos galima sunumeruoti bet kokia tvarka.
\end{itemize}

\TODO{\slide{33}}

\TODO{\slide{34}}

%Skaičiuojame taip, kaip yra sunumeruoti taškai. Nesvarbu, kad jie gali būti
%sunumeruoti kita tvarka.

\TODO{\slide{33-35}}

\TODO{\slide{36}}

\begin{prop}
  Jei $(n+1)$-osios eilės funkcijos $f(x)$ išvestinė intervale
  $\left[ x_{0}; x_{n} \right]$ yra apibrėžta:
  \begin{equation*}
    |f^{n+1}(x)| \leq M_{n+1}, \t{ kai } x_{0} < x < x_{n},
  \end{equation*}
  tai Niutono interpoliacinės formulės interpoliavimo paklaida įvertinama
  nelygybe:
  \begin{equation*}
    |f(x) - L_{n}(x)|
      \leq \frac{1}{(n+1)!}M_{n+1}|(x-x_{0})(x-x_{1})\cdots(x-x_{n}),
  \end{equation*}
  čia $x \in \left[ x_{0}; x_{n} \right]$.
  \begin{proof}
    \TODO{Įrodymas analogiškas tiesiniam atvejui, tik ši seka bus
    daug ilgesnė.}
  \end{proof}<++>
\end{prop}
\TODO{\slide{37}}
Išvados:
\begin{enumerate}
  \item Interpoliacinio daugianario laipsnį didinti tikslinga, kai:
    \begin{equation*}
      \frac{M_{n+1}}{(n+1)}|(x-x_{n})| \leq M_{n}.
    \end{equation*}
  \item Interpoliavimo paklaida priklauso nuo interpoliavimo mazgų
    $(x_{0},x_{1},\ldots,x_{n})$ išsidėstymo. Jie turi būti kuo arčiau
    interpoliavimo taško $x$ ir išsidėstę kuo simetriškiau jo atžvilgiu.
\end{enumerate}

\TODO{\slide{38}}

%$(2n - 1)!! = 1 \cdot 3 \cdot 5 \cdot \cdots \cdot (2n - 1)$
%$(2n)!! = 2 \cdot 4 \cdot 6 \cdots 2n$

\TODO{\slide{39}}

Niutono interpoliacinės formulės paklaida:
\begin{equation*}
  |f(x) - L_{n}(x)|
    \approx f(x,x_{0},x_{1},\ldots,x_{n})|(x-x_{0})(x-x_{1})\cdots(x-x_{n})|,
\end{equation*}
čia:
\begin{equation*}
  x \in \left[ \min_{i=0,\ldots,n}x_{i}; \max_{i=0,\ldots,n}x_{i} \right].
\end{equation*}
Kai atstumai tarp taškų maži (šiuo atveju išnaudojame ne visus taškus, o
pasiliekame vieną tašką paklaidos įvertinimui):
\begin{equation*}
  f(x,x_{0},x_{1},\ldots,x_{n})
    \approx \frac{f^{n+1}(x)}{(n+1)!}
    \approx f(x_{0},x_{1},\ldots,x_{n},x_{n+1})
\end{equation*}

\section{Ekstrapoliavimas}

\TODO{\slide{40}}

\begin{defn}[Ekstrapoliavimas]
  Funkcijos nežinomų reikšmių nustatymas taške, nepriklausančiame žinomų
  taškų intervalui.
\end{defn}

Interpoliavimą galimą taikyti tik vidinių taškų reikšmių
skaičiavimui. Išoriniams taškams reikia naudoti ektrapoliavimą.

\TODO{\slide{41}}

\TODO{\slide{42}}

Interpoliavimas netinka kai:
\begin{itemize}
  \item Duomenys su dideliu gradientu\footnote{Išvestinės vektorius.}
    (lygus grafikas su staigiu piku).
  \item Duomenys su triukšmu.
\end{itemize}
Tokiais atvejais atsiranda osciliacijos.

%TODO: Pasiieškoti Karlos Rungė. (Yra dar garsi fizikė Rungė.)

\TODO{\slide{43}}

\section{Lagranžo interpoliacinis daugianaris}

\TODO{\slide{44}}

Be Niutono interpoliacinio daugianario, dar naudojamas Lagranžo
interpoliacinis daugianaris. Lagranžo daugianaris iš esmės yra toks
pats kaip Niutono, tik jo nariai yra kitaip sugrupuoti. Lagranžo yra
patogesnis įrodymas, o praktikoje patogesnis yra Niutono.
\begin{align*}
  L_{n}(x)
    &= c_{0}(x)f(x_{0}) + \cdots + c_{n}(x)f(x_{n}) \\
    &= \sum_{i=0}^{n}c_{i}(x)f(x_{i}) \\
\end{align*}
Čia:
\begin{align*}
  c_{i}(x)
  &= \prod_{j=0,j \neq i}^{n}\frac{x-x_{j}}{x_{i}-x_{j}} \\
  &= \frac{P_{i}(x)}{P_{i}(x_{i})} \\
  &= \frac{(x-x_{0})\cdots(x-x_{i-1}(x-x_{i+1}))\cdots(x-x_{n})}
          {(x_{i}-x_{1})\cdots(x_{i}-x_{i-1})(x_{i}-x_{i+1})\cdots(x_{i}-x_{n})}
\end{align*}

\TODO{\slide{45-46}}

\TODO{\slide{47}}

Lagranžo interpoliavimas yra patogus, jei skaičiuojama naudojant tuos
pačius $x$ su skirtingais $y$ (tai yra, kai matavimai yra atliekami
tuose pačiuose taškuose), nes koeficientus $c_{k}(x)$ reikia
apskaičiuoti tik vieną kartą. Tačiau, jis yra mažiau patogus, jei
atsiranda papildomi duomenų taškai.

\TODO{\slide{48}}

\section{Atvirkštinis interpoliavimas}

\TODO{\slide{49}}

Žinomos $x$ ir $f(x)$ reikšmės:
\begin{itemize}
  \item Tiesioginis interpoliavimas leidžia apskaičiuoti $f(x)$ reikšmes
    gaunant naujas $x$ reikšmes.
  \item Atvirkštinis interpoliavimas leidžia apskaičiuoti $x$ reikšmes
    gaunant naujas $f(x)$ reikšmes. Kitaip tariant ieškome ne $f(x)$ kažkokiame
    taške $x$, o $x$ kuriame $f(x) = T$, kur $T$ yra kažkokia konstanta.
\end{itemize}

Atvirkštinio interpoliavimo idėjos:
\begin{enumerate}
  \item Sukeičiame vietomis $x$ ir $f(x)$ ir interpoliuojame. Bet
    netolygusis reikšmių $f(x_{i})$ pasiskirstymas dažnai duoda
    osciliuojantį interpoliacinį daugianarį.
  \item Sudorome $n$-ojo laipsnio interpoliacinį daugianarį
    aproksimuojantį pradinius duomenis $(x_{i},f(x_{i}))$, po to $x$
    randame spręsdami netiesinę lygtį. Šiuo atveju gali būti sąlygotas
    uždavinys.
\end{enumerate}

\TODO{\slide{50-52}}


%\section{3p}
%\cite[161]{textbook}
%\section{4p}
%\cite[162]{textbook}
%\section{7p}
%\cite[163]{textbook}
%\section{1T}
%\cite[161]{textbook}
%\section{9p}
%\cite[166]{textbook}
%\section{2T}
%\cite[167]{textbook}
%\section{1,2p}
%\cite[160]{textbook}
%\section{5,6p}
%\cite[163]{textbook}
%\section{8p}
%\cite[166]{textbook}
%\section{10p}
%\cite[167]{textbook}
%\section{11p}
%\cite[167]{textbook}
