\Chapter{Priedas}

\section{Veiksmai su matricomis}

\begin{defn}[Invertuojama matrica]
  Kvadratinė matrica $\matrixa$ yra vadinama invertuojama, jei egzistuoja
  tokia kvadratinė matrica $\matrixb$, kad tenkinama tapatybė:
  \begin{equation*}
    \matrixa \matrixb = \matrixb \matrixa = \matrixi
  \end{equation*}
  Jei matrica $\matrixb$ egzistuoja, tai ji yra vienareikšmiškai priklauso
  nuo matricos $\matrixa$ ir yra žymima $\matrixa^{-1}$.
\end{defn}

\begin{defn}[Transponuota matrica]
  Matricos $\matrixa$ transponuota matrica vadinama matrica $\matrixa^{T}$,
  kuri yra gaunama matricos $\matrixa$ elementus atspindėjus jos pagrindinės
  įstrižainės atžvilgiu. Matricos $\matrixa$ ir matricos
  $\matrixa^{T}$ elementai tenkina tokį sąryšį:
  \begin{equation*}
    \left[ \matrixa^{T} \right]_{ij} = \left[ \matrixa \right]_{ji}
  \end{equation*}
\end{defn}

\begin{defn}[Matricų sandauga]
  Matricos $\matrixa$, kurios dimensijos yra $n \times m$, ir matricos
  $\matrixb$, kurios dimensijos yra $m \times p$, sandauga
  $\matrixa\matrixb$ yra matrica, kurios dimensijos yra $n \times p$ ir
  kurios elementai yra lygūs:
  \begin{equation*}
    \left[ AB \right]_{ij}
      = \sum_{k=1}^{m} \left[ A \right]_{ik} \left[ B \right]_{kj}
  \end{equation*}
  Pavyzdžiui:
  \begin{equation*}
    \begin{pmatrix}
      a & b & c \\
      p & q & r \\
      u & v & w \\
    \end{pmatrix}
    \begin{pmatrix}
      x \\
      y \\
      z \\
    \end{pmatrix}
    =
    \begin{pmatrix}
      ax + by + cz \\
      px + qy + rz \\
      ux + vy + wz \\
    \end{pmatrix}
  \end{equation*}
\end{defn}
