\Chapter{Tiesinių lygčių sistemų sprendimas}

% 2014-02-18

\TODO{\slide{2-3}}

\TODO{\slide{4}}

Tiesinę lygčių sistemą:
\begin{equation*}
  \left\{
    \begin{array}{ccccccccc}
      a_{11}x_{1} &+& a_{12}x_{2} &+& \cdots &+& a_{1n}x_{n} &=& b_{1} \\
      a_{21}x_{1} &+& a_{22}x_{2} &+& \cdots &+& a_{2n}x_{n} &=& b_{2} \\
      \vdots & & \vdots & & \ddots & & \vdots &=& \vdots \\
      a_{n1}x_{1} &+& a_{n2}x_{2} &+& \cdots &+& a_{nn}x_{n} &=& b_{n} \\
    \end{array}
  \right.
\end{equation*}
patogu užrašyti matriciniu pavidalu:
\begin{equation*}
  \matrixa \vecx = \vecb
  \t{arba} 
  \begin{pmatrix}
    a_{11} & a_{12} & \cdots & a_{1n} \\
    a_{21} & a_{22} & \cdots & a_{2n} \\
    \vdots & \vdots & \ddots & \vdots \\
    a_{n1} & a_{n2} & \cdots & a_{nn} \\
  \end{pmatrix}
  \begin{pmatrix}
    x_{1} \\
    x_{2} \\
    \vdots \\
    x_{n} \\
  \end{pmatrix}
  =
  \begin{pmatrix}
    b_{1} \\
    b_{2} \\
    \vdots \\
    b_{n} \\
  \end{pmatrix}
\end{equation*}

\TODO{\slide{5}}
\begin{defn}[Įstrižainė matrica]
  \begin{equation*}
    \matrixd
    =
    \begin{pmatrix}
      d_{11}  & 0       & \cdots & 0 \\
      0       & d_{22}  & \cdots & 0 \\
      \vdots  & \vdots  & \ddots & \vdots \\
      0       & 0       & \cdots & d_{nn} \\
    \end{pmatrix}
  \end{equation*}
\end{defn}
\begin{defn}[Vienetinė matrica]
  \begin{equation*}
    \matrixi
    =
    \begin{pmatrix}
      1       & 0       & \cdots & 0 \\
      0       & 1       & \cdots & 0 \\
      \vdots  & \vdots  & \ddots & \vdots \\
      0       & 0       & \cdots & 1 \\
    \end{pmatrix}
  \end{equation*}
  \begin{equation*}
    \matrixa\matrixi = \matrixi\matrixa = \matrixa
  \end{equation*}
\end{defn}
\begin{defn}[Apatinė trikampė matrica]
  \begin{equation*}
    \matrixl
    =
    \begin{pmatrix}
      a_{11} & 0      & \cdots & 0 \\
      a_{21} & a_{22} & \cdots & 0 \\
      \vdots & \vdots & \ddots & \vdots \\
      a_{n1} & a_{n2} & \cdots & a_{nn} \\
    \end{pmatrix}
  \end{equation*}
\end{defn}
\begin{defn}[Viršutinė trikampė matrica]
  \begin{equation*}
    \matrixu
    =
    \begin{pmatrix}
      a_{11} & a_{12} & \cdots & a_{1n} \\
      0      & a_{22} & \cdots & a_{2n} \\
      \vdots & \vdots & \ddots & \vdots \\
      0      & 0      & \cdots & a_{nn} \\
    \end{pmatrix}
  \end{equation*}
\end{defn}
\TODO{\slide{6-7}}
\begin{defn}[Juostinė matrica]
  Matrica $\matrixa$ yra juostinė, jei $\exists r (r \in \NSET, r < n)$ toks, kad
  $a_{ij} = 0$, kai $|i-j|>r, i, j = 1,\ldots,n.$

  Kitaip tariant, išskyrus juostas prie pagrindinės įstrižainės (kurių plotis
  yra $2r + 1$), visi kiti elementai yra nuliniai.
\end{defn}
\begin{defn}[Triįstrižainė matrica]
  Juostinė matrica, kurios $r = 1$.
\end{defn}

\TODO{\slide{8}}
Kai lygčių sistemoje nedaug, ją galima lengvai išspręsti:
\begin{itemize}
  \item Grafiškai.
  \item Kramerio metodu.
  \item Kintamųjų eliminavimu.
\end{itemize}

Atkreipti dėmesį, kad:
\begin{itemize}
  \item Jei determinantas nelygus 0, tada yra lygiai vienas sprendinys.
  \item Jei determinantas lygus 0, tai matrica yra singualiari. Tokiu
    atveju sprendinių gali nebūti iš viso, arba jų gali būti be
    galo daug.
\end{itemize}

\TODO{\slide{9}}

Blogai sąlygotas uždavinys – jeigu dėl paklaidų, galima gauti visiškai
kitokią situaciją. (Pavyzdžiui, determinantas yra labai mažas skaičius.)

\TODO{\slide{10}}

\TODO{\slide{11}}
Kramerio metodas leidžia spręsti visas matricas, bet jis yra labai
lėtas.

\TODO{\slide{12}}
Tiesinių lygčių sprendimo metodai:
\begin{enumerate}
  \item Tiesioginiai metodai. Tinka spręsti lygtims, kurios turi iki $10^{4}$
    nežinomųjų. Tikslus sprendinys gaunamas per baigtinį žingsnių skaičių.
    \begin{itemize}
      \item Gauso.
      \item Skaidos.
      \item Choleckio.
      \item Perkelties.
    \end{itemize}
  \item Iteraciniai metodai. Tinka spręsti lygtims, kurios turi iki $10^{7}$
    nežinomųjų. Randamas apytikslis sprendinys bet kokiu norimu tikslumu.
    \begin{itemize}
      \item Jakobio.
      \item Zeidelio.
      \item Relaksacijos.
      \item Mišrusis.
      \item Variaciniai metodai – iteracinių metodų grupė tinkama spręsti
        lygtims turinčioms daugiau nei $10^{7}$ nežinomųjų. Variaciniais
        metodais yra ieškoma sprendinio keičiant uždavinio prigimtį
        (ieškomas minimumas hipererdvėje).
    \end{itemize}
\end{enumerate}
Skaidos, Choleckio, Perkelties metodai yra Gauso metodo modifikacijos.
Kadangi skaičiuojant kompiuteriu vis tiek gauname tik apytikslius
rezultatus, tai poreikis naudoti tiesioginius metodus mažėja.

Praktikoje paprastai yra naudojami dar sudėtingesni metodai, bet mes
jų nenaudosime.

\TODO{\slide{13}}
Pasirinkimas tarp tiesioginių ir iteracinių metodų gali priklausyti nuo
kelių faktorių:
\begin{itemize}
  \item Teorinis metodo efektyvumas.
  \item Matricos tipas. Pavyzdžiui, ar joje yra daug nulių, ar nenuliniai
    elementai yra išsidėstę tam tikra tvarka.
  \item Atminties laikymo reikalavimai.
  \item Kompiuterių architektūra.
\end{itemize}

\section{Tiesioginiai sprendimo metodai}

\TODO{\slide{14}}

Skaidos metodai naudojami kai daug kartų reikia spręsti tą pačią
lygtį su ta pačia matrica ir skirtingais $b_{i}$.

\subsection{Gauso metodas}

\cite[60-62]{textbook}

\TODO{\slide{15}}

Idėja:
\begin{itemize}
  \item Nuoseklus nežinomųjų šalinimas.
  \item Sistemos matricos pertvarkymas į viršutinę trikampę matricą.
  \item Sprendinys randamas iš pertvarkytosios matricos.
\end{itemize}

\TODO{\slide{16-20}}

\TODO{\slide{21}}
Gauso metodo skaičiavimo apimtis:
\begin{itemize}
  \item Tiesioginė eiga yra $O(\frac{2}{3}n^{3})$ aritmetinių veiksmų.
  \item Atbulinė eiga yra $O(\frac{1}{2}n^{2})$ aritmetinių veiksmų.
\end{itemize}
Daugiausiai laiko suryja tiesioginė eiga. Idėja: bandyti ją optimizuoti.

\TODO{\slide{22}}

\TODO{\slide{23}}


\TODO{\slide{24}}
Apvalinimo paklaidos:
\begin{itemize}
  \item Didelė dalis skaičiavimų su $\frac{1}{3}n^{3}$ operacijų.
  \item Svarbu – paklaida didėja.
  \item Didelėms sistemoms (virš 100 lygčių), apvalinimo paklaida gali
    būti pakankamai didelė.
  \item Blogai sąlygoti uždaviniai – maži koeficientų pokyčiai
    lemia didelius sprendinių pokyčius.
  \item Apvalinimo paklaidų analizė ypač svarbi blogai sąlygotiems
    uždaviniams.
\end{itemize}

\TODO{\slide{25}}

Jeigu matricos determinantą skaičiuojame tiesioginiu metodu, tai yra
labai brangi operacija. Jeigu panaudodami Gauso metodą suvedėme į
trikampę matricą, tai determinanto skaičiavimas smarkiai
supaprastėja:
\begin{equation*}
  \det A = \det U = a_{11} \tilde{a}_{22} \cdots \tilde{a}_{nn}.
\end{equation*}

\subsection{Pagrindinio elemento parinkimo būdai: 1, 2 ir 3}

\cite[66-68]{textbook}

\TODO{\slide{26}}
Gauso metodo galimi sunkumai:
\begin{itemize}
  \item Dalyba iš nulio.
  \item Apvalinimo paklaidos.
  \item Blogai sąlygoti uždaviniai.
\end{itemize}
Šias problemas galima bandyti spręsti parenkant pagrindinį elementą.
Parinkimas yra nereikalingas, jei:
\begin{itemize}
  \item Išpildyta pagrindinės įstrižainės vyravimo sąlyga:
    \begin{equation*}
      |a_{ii}| > \sum^{n}_{j=1,j \neq i}|a_{ji}|, i = 1,\ldots,n.
    \end{equation*}
  \item Matrica $\matrixa$ yra simetrinė:
    \begin{equation*}
      \matrixa^{T} = \matrixa
    \end{equation*}
    ir teigiamai apibrėžta:
    \begin{equation*}
      \forall \vecx \neq 0 : (\matrixa\vecx,\vecx) = \vecx^{T}\matrixa\vecx > 0,
    \end{equation*}
    čia $(\matrixa\vecx,\vecx)$ yra vektorių $\matrixa\vecx$ ir $\vecx$
    skaliarinė sandauga.
\end{itemize}

\TODO{\slide{27}}
Pagrindinio elemento parinkimo būdai:
\begin{enumerate}
  \item \emph{Iš stulpelio elementų.}
    Pagrindinis elementas parenkamas iš stulpelio elementų. Šios dvi lygtys
    sukeičiamos vietomis.
  \item \emph{Iš eilutės elementų.}
    Pagrindinis elementas parenkamas iš pertvarkomos eilutės elementų.
    Sukeičiami vietomis matricos $\matrixa$ stulpeliai ir įsimenama naujoji
    nežinomųjų tvarka.
  \item \emph{Pagal lygties koeficientų modulių sumą.}
    Kiekvienoje lygtyje randamas didžiausias koeficientas ir iš jo
    padalijama atitinkama lygtis. Lygtys sukeičiamos vietomis taip, kad
    būtų dirbama su lygtimi, kurios koeficientų modulių suma yra mažiausia.
    Nežinomieji pernumeruojami taip, kad būtų dalinama iš didžiausio
    koeficiento.
\end{enumerate}
Pagrindinio elemento parinkimas pagal koeficientų modulių sumą yra
naudingas kai lygtyje yra daug skirtingos prigimties koeficientų (labai
mažų ir labai didelių).

\TODO{\slide{28-37}}

\subsection{Triįstrižainės sistemos (perkelties algoritmas)}

\cite[72-73]{textbook}

\TODO{\slide{38}}

Triįstrižainė matrica yra matrica turinti tris nenulines įstrižaines.

\TODO{\slide{39}}

\TODO{\slide{40}}
Perkelties metodo algoritmas \en{Thomas algorithm, tridiagonal matrix
algorithm}:
\begin{enumerate}
  \item Tiesioginė eiga.
    \begin{align*}
      C_{1} &= -\frac{c_{1}}{b_{1}} \\
      D_{1} &= \frac{d_{1}}{b_{1}} \\
      C_{k} &= -\frac{c_{k}}{a_{k}C_{k-1}+b_{k}}, k=2,3,\ldots,n-1 \\
      D_{k} &= \frac{d_{k}-a_{k}D_{k-1}}{a_{k}C_{k-1}+b_{k}}, k=2,3,\ldots,n \\
    \end{align*}
  \item Atbulinė eiga.
    \begin{align*}
      x_{n} &= D_{n} \\
      x_{k} &= C_{k}x_{k+1} + D_{k}, k=n-1,n-2,\ldots,1 \\
    \end{align*}
\end{enumerate}

\TODO{\slide{41}}
\begin{prop}[Perkelties metodo pakankama konvergavimo sąlyga]
  Jei
  \begin{equation*}
    |b_{i}| \geq |a_{i}| + |c_{i}|, i=1,\ldots,n
  \end{equation*}
  ir bent su vienu $i$ galioja griežta nelygybė, tai dalyba iš nulio ar
  labai mažo skaičiaus perkelties metodo eigoje negalima.
  \note{Perkelties metodo konvergavimo sąlyga skiriasi nuo Gauso metodo sąlygos
  tuo, kad Gauso metodo sąlygoje yra griežta nelygybė.}
\end{prop}

\TODO{\slide{42-44}}

\subsection{Skaidos metodas}

\cite[76-78]{textbook}

\TODO{\slide{45-49}}

\TODO{\slide{50}}
\begin{prop}
  Visi matricos $\matrixa$ pagrindiniai minorai\footnote{Determinantai
  prie įstrižainės.} nelygūs nuliai tada ir tik tada kai egzistuoja
  apatinė trikampė matrica
  $\matrixl (l_{ii}=1, \forall i)$ ir viršutinė trikampė matrica
  $\matrixu$ tokios, kad $\matrixa = \matrixl \matrixu$.
\end{prop}
\note{Išskaidymo metodas dar kitaip yra vadinamas faktorizacijos metodu.}

\TODO{\slide{51}}

\TODO{\slide{52}}
Skaidos metodo žingsniai:
\begin{enumerate}
  \item Išskaidome $\matrixa$ į $\matrixl$ ir $\matrixu$ sandaugą.
  \item Žinodami $\vecb$ randame $\vecd$ iš $\matrixl\vecd = \vecb$.
  \item Spręsdami $\matrixu\vecx = \vecd$ apskaičiuojame $\vecx$.
\end{enumerate}
Sklaidos metodo pranašumas yra tai, kad vieną kartą apskaičiavus $\matrixl$ ir
$\matrixu$ galima spręsti sistemas su skirtingais $\vecb$ nekartojant
matricos $\matrixa$ išskaidymo.

\TODO{\slide{53-54}}

\TODO{\slide{55}}
LU metodo algoritmas:
\begin{enumerate}
  \item \TODO{Išskaidymas \slide{54}}
  \item Tiesioginė eiga (keitimas):
    \begin{equation*}
      d_{i} = b_{i} - \sum_{j=1}^{i-1}l_{ij}d_{j}, i=1,\ldots,n
    \end{equation*}
  \item Atbulinė eiga (kaip ir Gauso metode):
    \begin{align*}
      x_{n} &= \frac{d_{n}}{u_{nn}} \\
      x_{i} &= \frac{d_{i} - \sum_{j=i+1}^{n}u_{ij}x_{j}}{u_{ii}},
        i=n-1,\ldots,2,1. \\
    \end{align*}
\end{enumerate}

\TODO{\slide{56-64}}

\TODO{\slide{65}}
Skaidos metodas turi visas tas pačias problemas kaip ir Gauso, todėl
Skaidos metodui irgi yra taikomas pagrindinio elemento parinkimas.

\TODO{\slide{66}}

\subsection{Choleckio metodas ir algoritmas}

\cite[79-80]{textbook} \cite[81]{textbook}.

\TODO{\slide{67}}
Jei matrica $\matrixa$ yra simetrinė ir teigiamai apibrėžta\footnote{Matrica
yra teigiamai apibrėžta, kai visos tikrinės reikšmės yra teigiamos.}, tai
tokiu atveju yra patogu naudoti Choleckio dekompoziciją:
\begin{equation*}
  \matrixa = \matrixl\matrixl^{T} = \matrixu^{T}\matrixu
\end{equation*}
\note{Kai matrica $\matrixa$ yra simetrinė ir teigiamai apibrėžta, pagrindinio
elemento parinkimas nereikalingas.}

\TODO{\slide{68-69}}

\TODO{\slide{70}}
Choleckio metodo rekurentinės formulės:
\begin{align*}
  u_{ii} &= \sqrt{a_{ii} - \sum_{k=1}^{i-1}u_{ki}^{2}} \\
  u_{ij} &= \frac{a_{ij} - \sum_{k=1}^{i-1}u_{ki}u_{kj}}{u_{ii}},
    j=i+1,\ldots,n \\
\end{align*}

%\section{Gauso-Žordano metodas}
%\cite[63]{textbook}
%\section{Nesuderintos sistemos}
%\cite[65]{textbook}
%\section{Gauso metodo skaičiavimų apimtis}
%\cite[69-70]{textbook}
%\section{8p}
%\cite[77]{textbook}
