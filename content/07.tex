\Chapter{Netiesinių lygčių sistemų sprendimas}

% 2014-03-18

\TODO{\slide{2}}

%Šiandien šiek tiek dar per anksti kalbėti apie netiesinių lygčių
%sprendimą, nes jos yra sprendžiamos iteraciniais metodais, o mes dar
%neturėjom tiesinių lygčių sprendimo iteraciniais metodais.

\section{Vektorių ir matricų normos}

\TODO{\slide{3}}

\TODO{\slide{4}}

\begin{defn}[Vektoriaus norma]
  Vektoriaus $\vecx$ norma (žymima $||\vecx||$) vadinamas skaičius
  tenkinantis tokias savybes:
  \begin{enumerate}
    \item $||\vecx|| > 0$, jei $\vecx \neq 0$, o nulinio vektoriaus
      norma $||\veczero|| = 0$.
    \item $||\alpha \vecx|| = |\alpha|\cdot||\vecx||$, čia $\alpha$ yra
      skaičius.
    \item $||\vecx + \vecy|| \leq ||\vecx|| + ||\vecy||$ bet kuriems
      $\vecx$ ir $\vecy$.
  \end{enumerate}
\end{defn}

%Galime sugalvoti bet kokią normą, svarbu tik kad ji tenkintų normos
%reikalavimus. Ypač naudinga norint ką nors įrodyti.

\TODO{\slide{5}}

Dažniausiai taikomos vektoriaus $\vecx$ normos:
\begin{enumerate}
  \item Modulių sumos:
    \begin{equation*}
      ||\vecx||_{1} = \sum_{k=1}^{N}|x_{k}|.
    \end{equation*}
  \item Kvadratinė:
    \begin{equation*}
      ||\vecx||_{2} = \sqrt{\sum_{k=1}^{N}|x_{k}|^{2}}
    \end{equation*}
  \item Maksimumo:
    \begin{equation*}
      ||\vecx||_{\infty} = \max_{1 \leq k \leq N} |x_{k}|.
    \end{equation*}
\end{enumerate}

%Pavyzdys: koks atstumas tarp Naugarduko ir Baltupių. Klausimas: kokią normą
%naudojame (oro kelią, visuomeninį transportą, Manhetano atstumas).

\TODO{\slide{6}}

Minėtos vektoriaus normos yra $p$-normos, apibrėžtos kaip
\begin{equation*}
  ||\vecx||_{p} = \left( \sum_{k=1}^{N}|x_{k}|^{p} \right)^{\frac{1}{p}}
\end{equation*}
atskirieji atvejai.

\TODO{\slide{7}}

\begin{defn}[Matricos norma]
  Matricos $\matrixa$ norma (žymima $||\matrixa||$) vadinamas skaičius
  turintis šias savybes:
  \begin{enumerate}
    \item $||\matrixa|| > 0$, jei $\matrixa \neq 0$, o nulinės matricos
      norma $||\matrixzero|| = 0$.
    \item $||\alpha\matrixa|| = |\alpha|\cdot||\matrixa||$, čia $\alpha$
      yra skaičius.
    \item $||\matrixa + \matrixb|| \leq ||\matrixa|| + ||\matrixb||$
      bet kurioms $\matrixa$ ir $\matrixb$.
    \item $||\matrixa\matrixb|| \leq ||\matrixa|| \cdot ||\matrixb||$
      bet kurioms suderintoms $\matrixa$ ir $\matrixb$.
  \end{enumerate}
\end{defn}

\TODO{\slide{8}}
\begin{defn}[Suderintos normos]
  Vektoriaus $\vecx$ norma yra suderinta su matricos $\matrixa$ norma
  jei:
  \begin{equation*}
    ||\matrixa\vecx|| \leq ||\matrixa|| \cdot ||\vecx||.
  \end{equation*}
\end{defn}

Normos yra suderintos, kai matricos $\matrixa$ norma yra apibrėžiama
panaudojant vektoriaus normą:
\begin{equation*}
  ||\matrixa|| = \max_{x \neq 0} \frac{||\matrixa\vecx||}{||\vecx||}.
\end{equation*}
Norma aprašo didžiausią vektoriaus pailgėjimą, atvaizduojant operatoriumi
$\matrixa$.

\TODO{\slide{9}}
Matricų normos (suderintos su vektorių normomis):
\begin{enumerate}
  \item Stulpelių modulių suma ($p=1$):
    \begin{equation*}
      ||\matrixa||_{1} = \max_{1 \leq j \leq N} \sum_{i=1}^{N} |a_{ij}|.
    \end{equation*}
  \item Kvadratinė ($p=2$):
    \begin{equation*}
      ||\matrixa||_{2}
        = \sqrt{\max_{1 \neq i \neq N} |\lambda_{i}(\matrixa^{T}\matrixa)|}.
    \end{equation*}
  \item Eilučių modulių suma ($p=\infty$):
    \begin{equation*}
      ||\matrixa||_{\infty} = \max_{1 \leq i \leq N} \sum_{j=1}^{N}|a_{ij}|.
    \end{equation*}
\end{enumerate}

\TODO{\slide{10-11}}
\TODO{\slide{12}}

%$A^{T}A$ – yra simetrinė matrica.

%Tikrinės reikšmės:
%$Ax = \lambda x$, čia x – tikrinis vektorius, o $\lambda$ – tikrinė reikšmė.
%$x \neq 0$, $\lambda$ – skaičius.

%\TODO{\eng{eigenvalue}, \eng{eigenvector} Pasižiūrėti TASM paskaitas.}

%$(e_{i}, e_{j}) = 0, i \neq j$ 
%$||e|| = (e_{i}, e_{i}) = 1$


%Normų žymėjimas: $|| \t{kažkas}}||_{\t{normos indeksas}}$. Jei indekso nėra,
%tai kalbama apie kvadratines normas.

\TODO{\slide{13}}
\TODO{\slide{14}}

%$(Ax, Ax)$ – skaliarinė santauga.

%Tikrinės reikšmės – labai dažnai gali būti naudojamos
%charakterizuoti, parodyti kas yra gerai, o kas blogai.

\TODO{\slide{15}}

\begin{defn}[Didesnė simetrinė matrica]
  Simetrinė matrica $\matrixa$ yra didesnė arba lygi simetrinei matricai
  $\matrixb$ ($\matrixa \geq \matrixb$), jei kiekvienam vektoriui
  $\vecx$ galioja skaliarinių sandaugų nelygybė:
  \begin{equation*}
    (\matrixa\vecx, \vecx) \geq (\matrixb\vecx, \vecx).
  \end{equation*}
\end{defn}

\section{Netiesinių lygčių sistemų sprendimas}

\TODO{\slide{16}}

%Reikia apjungti žinias iš tiesinių lygčių sistemų ir netiesinių
%lygčių sprendimo.

\TODO{\slide{17}}

\TODO{\slide{18}}

Skiriasi nuo vienmačio atveju tik tuo, kad dirbame su vektoriais.
Paprastųjų iteracijų metodas \en{Fixed-point method}:
\begin{equation*}
  f(\vecx) = \veczero \iff \vecx = \vecS(\vecx)
\end{equation*}
Žinomas $k$-asis vektoriaus $\vecx$ artinys $\vecx^{k}$, iš to gauname
iteracinį procesą:
\begin{equation*}
  x^{k+1} = \vecS(x^{k})
\end{equation*}

\TODO{\slide{19}}

\begin{defn}[Spūdinė funkcija]
  Funkcija $\vecS(\vecx)$ vadinasi spūdine funkcija $n$-mačių vektorių
  aibėje $V$, jei egzistuoja toks skaičius $q \in \left( 0; 1 \right)$, kad
  \begin{equation*}
    \forall \vecx, \vecy \in V :
      ||\vecS(\vecx) - \vecS(\vecx)|| \leq q||\vecx - \vecy||.
  \end{equation*}
  \note{Spūdinė funkcija yra tolydžioji.}
\end{defn}

%Mums reikia, kad iteracinis procesas konverguotų.

%Pavyzdys: jei intervalas būtų jau nuo 1 iki 10, tai funkcija $S(x) = x^{3}$
%jau nebebūtų spūdinė.

\TODO{\slide{20}}

\begin{prop}[Paprastųjų iteracijų metodo konvergavimo sąlygos]
  Sakykime, kad vektoriaus $\veca$ aplinkoje
  $\Omega_{\delta}(\veca) = \left\{ \vecx :
    || \vecx - \veca || \leq \delta \right\}$
  yra apibrėžta spūdinė funkcija $\vecS(\vecx)$ ir
  $||\vecS(\veca) - \veca|| \leq (1 - q)\delta$.
  Tada aplinkoje $\Omega_{\delta}(\veca)$:
  \begin{enumerate}
    \item Yra vienintelis vektorinės lygties $\vecx = \vecS(\vecx)$ sprendinys
      $\vecx*$.
    \item Paprastųjų iteracijų seka $\vecx^{k+1} = \vecS(\vecx^{k})$ su bet
      kokiu pradiniu artiniu $\vecx^{0} \in \Omega_{\delta}(\veca)$ konverguoja
      į šį sprendinį.
    \item $k$-ojo artinio paklaida įvertinama nelygybe:
      \begin{equation*}
        ||\vecS(\vecx^{k}) - \vecS(\vecx*)||
          \leq \frac{q^{k}}{1-q}||\vecS(\vecx^{0}) - \vecx^{0}||.
      \end{equation*}
  \end{enumerate}
\end{prop}

%$\delta$ – spindulys $n$-matėje erdvėje.

%$q$ yra mažas skaičius, nuo 0 iki 1.

%$a$ yra pagalbinis vektorius.

\TODO{\slide{21-22}}

\TODO{\slide{23}}

Paprastųjų iteracijų metodo privalumai:
\begin{itemize}
  \item Nereikia skaičiuoti dalinių išvestinių.
  \item Tiesinis konvergavimo greitis.
  \item Netiesinių lygčių sistema perrašoma nevienareikšmiškai:
    \begin{equation*}
      f(\vecx) = 0 \iff \vecx = \vecS(\vecx).
    \end{equation*}
    Svarbu, kad:
    \begin{itemize}
      \item $\vecS(\vecx)$ tenkintų konvergavimo sąlygas.
      \item Konstanta $q$ būtų kuo mažesnė.
    \end{itemize}
\end{itemize}

%\section{Paprastųjų iteracijų metodas}

\TODO{\slide{24}}

\begin{defn}[Išreikštinis iteracinis metodas]
  Išreikštinis iteracinis metodas skirtas spręsti lygtį $f(\vecx) = 0$:
  \begin{equation*}
    \frac{\vecx^{k+1} - \vecx^{k}}{\tau} = f(\vecx^{k}),
  \end{equation*}
  čia $\tau$ yra iteracinis parametras, parenkamas taip, kad iteracinis
  procesas konverguotų greičiausiai. Šiuo atveju iteracinio metodo
  funkcija būtų:
  \begin{equation*}
    \vecS(\vecx) = \vecx + \tau f(\vecx).
  \end{equation*}
\end{defn}

\begin{defn}[Pikaro metodas]
  Pikaro metodas skirtas spręsti lygtį $f(\vecx) = 0$. Tegul
  \begin{equation*}
    f(\vecx) = \matrixa \vecx + G(\vecx),
  \end{equation*}
  čia $\matrixa$ yra $n$-osios eilės matrica, tiesinė funkcijos dalis, o
  $G(\vecx)$ yra netiesinė dalis. Iteracinis procesas:
  \begin{equation*}
    \matrixa \vecx^{k+1} + G(\vecx^{k}) = 0
  \end{equation*}
  konverguoja, jei $\vecS(\vecx) = \matrixa^{-1}G(\vecx)$ yra spūdinė
  funkcija.
\end{defn}

\section{Niutono metodas}

\TODO{Sustota: 6BP\_SM8.pdf, 25 skaidrė; konspektai/6\_paskaita.tex}
