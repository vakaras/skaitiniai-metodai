<| block chapters |>
\newcommand{\cinput}[1]{\input{../content/#1}}

\cinput{01.tex}   % Kompiuterių aritmetika ir algoritmai
\cinput{02.tex}   % Netiesinių lygčių sprendimas
\cinput{03.tex}   % Tiesinių lygčių sistemų sprendimas
\cinput{04.tex}   % Interpoliavimas

%\cinput{05.tex}
\Chapter{Splainai}
% 2014-03-04
\section{Splainų interpoliavimas}
\cite[170-173]{textbook}
%\section{12p}
%\cite[171]{textbook}
\section{Kubinis splainas}
\cite[174-175]{textbook}

%\cinput{06.tex}
\Chapter{Mažiausių kvadratų metodas}
% 2014-03-18
% K.Plukas Skaičiavimo metodai ir algoritmai Kaunas Naujasis lankas, 2001.
\Chapter{Spūdusis atvaizdis}
\section{Spūdusis atvaizdis ir jo savybės}
\cite[43-44]{textbook}
\section{Vektoriai ir matricos}
\section{Vektoriaus norma, matricos norma, suderintos normos}
\cite[43-44]{textbook}
\cite[104]{textbook}
%\section{Pratimas 2}
%\cite[113]{textbook}
\section{Simetrinė matrica, jos norma}
\section{Nelygybės tarp simetrinių matricų}
\cite[99]{textbook}
\section{Netiesinių lygčių sistemų sprendimas}
\cite[43-48]{textbook}
%\section{7p}
%\cite[46-47]{textbook}

%\cinput{07.tex}
\Chapter{Iteraciniai metodai}
% 2014-03-25
\section{Iteracinių metodų klasifikacija}
\cite[104]{textbook}
\section{Jakobio metodas}
\cite[88-93]{textbook}
\section{Zeidelio metodas}
\cite[93-96]{textbook}
\section{Relaksacijos metodas}
\cite[96-100]{textbook}
\section{Neišreikštiniai stacionarieji iteraciniai metodai}
\cite[110-112]{textbook}
\section{Iteracinių metodų klasifikacija}
\cite[104]{textbook}
\section{Stacionariųjų metodų pakankamoji konvergavimo sąlyga}
\cite[105]{textbook}
%\section{T}
%\cite[105]{textbook}
%\section{Pratimas 3}
%\cite[113]{textbook}
\section{Optimali iteracinio parametro reikšmė}
\cite[106]{textbook}
\section{Sąlygotumo skaičius}
\cite[107]{textbook}
\section{Neišreikštinių stacionariųjų metodų sąlygos}
\cite[109]{textbook}

%\cinput{08.tex}
\Chapter{Variaciniai metodai}
% 2014-04-01
%\section{t}
%\cite[114]{textbook}
\section{Didžiausio nuolydžio}
\cite[115-118]{textbook}
\section{Jungtinių gradientų metodas}
\cite[120-121]{textbook}

%\cinput{09.tex}
\Chapter{Tikrinių reikšmių uždavinys}
% 2014-04-08
\section{Matricos tikrinės reikšmės ir vektoriai}
%\section{1p}
%\cite[133]{textbook}
\section{Pilnoji vektorių sistema}
\cite[134]{textbook}
\section{Šmidto ortogonalizavimo algoritmas}
\cite[201]{textbook}
\section{Charakteristinio daugianario reikšmių radimo rekurentinė
formulė triįstrižaininėms matricoms}
\cite[135]{textbook}
%\section{2p}
%\cite[136]{textbook}
\section{Parabolių metodas}
\cite[136-137]{textbook}
%\section{3p}
%\cite[137-138]{textbook}
%\section{4p}
%\cite[139-141]{textbook}
% 2014-04-15
\section{Šturmo grandinė ir Geršgorino teorema}
\cite[144]{textbook}
%\section{6p}
%\cite[145]{textbook}
\section{Laipsnių metodas}
\cite[151-154]{textbook}
%\section{9p}
%\cite[153]{textbook}
%\section{10p}
%\cite[154]{textbook}
\section{Atvirkštinių iteracijų metodas}
\cite[141-142,146-148]{textbook}
%\section{5p}
%\cite[142]{textbook}
%\section{7p}
%\cite[147]{textbook}
\section{Hausholderio transformacija}
\cite[148-151]{textbook}

%\cinput{10.tex}
\Chapter{Skaitinis integravimas}
% 2014-04-29, 2014-05-06
\section{Stačiakampių formulės}
\cite[181-184]{textbook}
\section{Trapecijų formulė}
\cite[184-186]{textbook}
\section{Simpsono formulė}
\cite[187-188]{textbook}
\section{Paklaidos įvertinimo būdai}
\cite[189-191]{textbook}
\section{Rungės taisyklė}
\cite[190-192]{textbook}
\section{Neapibrėžtinių koeficientų metodas}
\cite[195-197]{textbook}
\section{Adaptyvieji skaitinio integravimo metodai}
\cite[193-194]{textbook}
\section{Gauso skaitinio integravimo formulės}
\cite[200-204]{textbook}
%\section{7p}
%\cite[201]{textbook}
%\section{8,9p}
%\cite[205]{textbook}

%\cinput{11.tex}
\Chapter{Skaitinis diferencialinių lygčių sprendimas}
% 2014-05-13
\cite[9-42]{textbook-differential-equations}
\section{Koši uždavinio sprendimas}
\section{Eulerio metodas}
\section{Rungės ir Kutos metodas}
\section{Aproksimavimo paklaidos įvertinimas}
\section{Metodo stabilumas ir konvergavimas}

%\cinput{12.tex}
\Chapter{Funkcijų optimizavimo metodai}
% 2014-05-20
\cite[211-213]{textbook}
\section{Aukso pjūvio metodas}
\cite[213-215]{textbook}
%\section{1,2p}
%\cite[216-217]{textbook}
\section{Niutono metodas}
\cite[217]{textbook}
%\section{3p}
%\cite[218]{textbook}
%\section{4p}
%\cite[219]{textbook}
\section{Kelių kintamųjų funkcijų minimizavimas}
\cite[220-221]{textbook}
\section{Simpleksų metodas}
\cite[222-224]{textbook}
\section{Simpleksų metodo algoritmas}
\cite[225]{textbook}
\section{Gradientų metodas}
%\section{8p}
%\cite[226-228]{textbook}

\cinput{appendix.tex}
\cinput{changes.tex}
<| endblock |>
